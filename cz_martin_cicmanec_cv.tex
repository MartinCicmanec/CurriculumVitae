\documentclass[8pt]{article}
\usepackage{array, xcolor, bibentry}
\usepackage[margin=1cm]{geometry}
\usepackage{hyperref}
\usepackage{ragged2e}

\pdfinfo{
   /Author (Martin Čičmanec)
   /Title  (Curriculum Vitae)
   /CreationDate (D:20230430195600)
   /Subject (CV for job applications)
   /Keywords (PDF;LaTeX;CV;IT;Python;Linux)
}

\title{\bfseries\Large CV - Martin Čičmanec}
\author{martin.cicmanec.47@gmail.com}
\date{}

\definecolor{lightgray}{gray}{0.8}
\newcolumntype{L}{>{\raggedleft}p{0.14\textwidth}}
\newcolumntype{R}{p{0.9\textwidth}}
\newcommand\VRule{\color{lightgray}\vrule width 0.5pt}

\begin{document}
\maketitle
\begin{minipage}[ht]{0.48\textwidth}
    Pod Drinopolem 1660/2\\
    Praha 6\\
    Czech Republic
\end{minipage}
\begin{minipage}[ht]{0.48\textwidth}
    \raggedleft Praha\\
    \date{30.04.2023}\\
    +421 948 725 688
\end{minipage}
\vspace{10pt}

\section*{Vzdělání}
\begin{tabular}{L!{\VRule}R}
    2010 -- 2013&(Nedokončeno) MFF UK, Fyzika - Nevhodný výběr oboru. Objevil jsem svůj zájem o IT.\\
    2013 -- 2016&(Nedokončeno) ČVUT FEL, Otevřená Informatika - Znalosti, které jsem získal během\\
        &svého studia mi poskytly neocenitelné zkušenosti, které jsem využil ke svému profesnímu\\
        &i osobnímu růstu. Tyto zkušenosti mě přesvědčily, že pro mě má praxe větší přínos než\\
        &teoretické studium, a proto jsem se rozhodl studium ukončit a věnovat se praxi a samostudiu.\\
\end{tabular}

\section*{Profesionální zkušenosti}
\begin{tabular}{L!{\VRule}R}
    4.2015 -- 6.2016 &{\bf Junior JAVA Developer ve firmě Certicon.}\\
    &Brigáda popři studiu. Převažně JAVA programování.\\
    &Práce na komunikačním JAVA modulu pro PLC TECO.\\
    &Vývoj multiagentního systému JAVA JADE pro vyjednávaní optimálního odběru energie domácností.\\
    &JAVA projekt pro zjednodušení a automatizaci psaní skriptů simulačního nástroje GridSim.\\[5pt]
    11.2016 -- today&{\bf Python vývojář ve firmě Optisolutions s.r.o.}\\
    &Mnoho zkušeností s Python, OpenCV, NumPy, akvizice fotek, inference v Torch.\\
    &Spolupráce s PostgreSQL databázemi a web aplikacemi.\\
    &Linux, Bash - správa a nastavení systémů, nasazovaní v produkci, virtualizace Qemu/KVM.\\
    &Docker - vývoj multikontejnerových aplikací.\\
    &Universal Robots programování robotů, jejich integrace do ROS prostrědí.\\
    &Vývoj řídícího Python ROS modulu a simulace projektu v Gazebo.\\
    &Siemens TIA Portal programování a automatizace. Práce na systémech S7-1500, S7-1200 a LOGO!.\\
    &Komunikace PLC-PLC a PLC-PC přes Profinet, I-Device, Profibus, TCP, Snap7, OPC-UA, ADS.\\
    &Programování víceosých pohonů Sinamics, Simotion a Rexroth.\\[5pt]
\end{tabular}

\section*{Zajímavé předchozí projekty}
\hfill\begin{minipage}{\dimexpr\textwidth-2em}
    Vision system for Production Cell 4.0 (\href{https://youtu.be/bp4fXCKxH9c}{https://youtu.be/bp4fXCKxH9c})\\
    Magic Eye preventative maintenance (\href{https://youtu.be/OJSRanzvh8g}{https://youtu.be/OJSRanzvh8g})\\
\end{minipage}

\section*{Jazyky}
\begin{tabular}{L!{\VRule}R}
    Slovenčina&rodný jazyk\\
    Čeština&samouk\\
    English&B2\\
    Němčina&B1\\
\end{tabular}

\section*{Shrnutí}
\hfill\begin{minipage}{\dimexpr\textwidth-2em}
    Více než šest let jsem pracoval na mnohých různorodých projektech, od průmyslové automatizace,\\
    přes řízení šestiosých robotů až po návrh a implementaci Python EdgePC systému běžícím v Docker kontejnerech.\\
    Během své práce jsem se často setkával s novými technologiemi a musel jsem je za krátkou dobu ovládnout a aplikovat.\\
\end{minipage}

\pagenumbering{gobble}

\end{document}